\section{Outline of Final Paper}
\label{sec:outline}

\begin{outline}[enumerate]
\1 Background
\2 How Bcfg2 enforces updates
\3 Why Bcfg2 enforces updates

\1 The Problem
\2 Automatic updates on most (but not all) packages
\2 Manual updates on more ``impactful'' packages
\3 How ``impactful'' packages are selected
\2 Different update policies for different systems
\3 Some systems need to peg to a certain package set
\2 Need to maintain separate repositories for development, test, and
production.
\2 Must be auditable and enforce privilege separation

\1 Other solutions -- why they won't work
\2 Yum excludes: difficult to push out updates of ``impactful'' packages
\2 yum-versionlock: extremely tedious to push out updates; error-prone
\2 Specifying versions in Bcfg2: fairly tedious; error-prone;
transitional problems in environment with multiple configuration
management tools
\2 Yum priorities: not a real solution to the problem, but part of the
solution
\2 Spacewalk: PostgreSQL version is very underfeatured; feature creep; lack
of support for SLES

\1 Pulp as a solution
\2 Automatic sync with blacklist and/or whitelist of packages,
repositories, and distros
\2 Manual promote/demote
\2 Logs for auditability
\2 Enforces distro-wide consistency
\2 Auditability of package errata

\1 Future development
\2 Tiered repositories
\2 Guaranteeing a minimum package age
\end{outline}
