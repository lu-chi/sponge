\section{The Problem}

The problem we set out to solve was one familiar to many systems
administrators: a sane approach to installing package updates.
Namely, we sought to find a balance between maintaining systems full
up-to-date in order to ensure that security fixes are applied in a
timely manner; and the need to test packages that might be high-risk
or ``impactful'' before deploying them in order to avoid unplanned
outages.  Beattie et al. observed that up to 18\% of initial patches
to resolve public vulnerabilities were faulty on the day of their
release, a number that dropped to 6\% by 30 days after the release of
the initial patch~\cite{BACWWS02}.

Moreover, we sought to do this as automatically as possible.

In a large, heterogenous environment such as ours -- over a thousand
servers, over 20 admins in 5 different teams, managing three projects
often with very different functional requirements -- it was necessary
that any approach be very flexible as well.  We require the ability to
maintain different update policies for different systems, including
the selection of different ``impactful'' packagesets: while the
``postfix'' package might be quite high-risk for a mail server, it's
trivial on a compute cluster node, for instance.  Some systems are so
sensitive to package versions that they needed to be pegged at a
specific set of packages -- no updates at all.  (As we shall see, this
is not always a simple task.)

Moreover, in addition to maintaining different update policies and
``impactful'' package sets for different types of machines, we needed
to maintain separate repositories -- with different package versions
-- for development, test, and production environments.

Finally, any solution we selected need to be auditable and enforce
reasonable privilege separation.  With five different teams working on
very different tasks, it was important that we not be able to
``clobber'' each other's work.
